\draw [draw=none, fill=red] (90:1.5) circle (2cm);
\draw [draw=none, fill=green] (-30:1.5) circle (2cm);
\draw [draw=none, fill=blue] (210:1.5) circle (2cm);

% Draw areas where two of the three primary colors are overlapping.
% These areas are the secondary colors yellow, cyan and magenta.
\begin{scope} % red + green = yellow
	\clip (90:1.5) circle(2cm);
	\draw [draw=none, fill=yellow] (-30:1.5) circle (2cm);
\end{scope} % blue + red = magenta
\begin{scope}
	\clip (210:1.5) circle(2cm);
	\draw [draw=none, fill=magenta] (90:1.5) circle (2cm);
\end{scope}
\begin{scope} % green + blue = cyan
	\clip (-30:1.5) circle(2cm);
	\draw [draw=none, fill=cyan] (210:1.5) circle (2cm);
\end{scope}

% Draw the center area which consists of all the primary colors.
\begin{scope} % red + green + blue = white
	\clip (90:1.5) circle(2cm);
	\clip (210:1.5) circle(2cm);
	\draw [draw=none, fill=white] (-30:1.5) circle (2cm);	
\end{scope}
